\beginoef

Gegeven onderstaande code.
\begin{footnotesize}
\begin{verbatim}
#include <memory>
#include <iostream>
using namespace std;

void schrijf(const string * s, int aantal){
    cout<<endl;
    for(int i=0; i<aantal-1; i++){
        cout<<s[i]<<" - ";
    }
    cout<<s[aantal-1];
}

void verwijder(string * s, int aantal, int volgnr){
    if(volgnr < aantal){
        for(int i = volgnr; i < aantal-1; i++){
            s[i] = s[i+1];	
        }
    }
}

int main(){
	
    string namen[]={"Rein","Ada","Eppo"};
	
    schrijf(namen,3);
    verwijder(namen,3,0);
    schrijf(namen,3);
	
    return 0;
}
\end{verbatim}
\end{footnotesize}
Je weet dat de regel code \verb}s[i] = s[i+1];} het kopi\"eren van een string impliceert.
Dat moeten we vermijden, want een string kan in principe heel groot zijn. 
Schrijf twee nieuwe procedures, die bij het onderstaande hoofdprogramma horen.
We bewaren nu (unique) pointers in de array, zodat we bij het opschuiven van de elementen in de array
enkel pointers moeten verleggen, en geen kopie\"en maken.
\begin{footnotesize}
\begin{verbatim}
int main(){
    unique_ptr<string> pnamen[]={make_unique<string>("Rein"),
                                 make_unique<string>("Ada"), 
                                 make_unique<string>("Eppo")};	 
    schrijf(pnamen,3);
    verwijder(pnamen,3,0);
    schrijf(pnamen,3);
	
    return 0;
}
\end{verbatim}
\end{footnotesize}
Probeer ook eens het laatste element uit de array te verwijderen!

\endoef
