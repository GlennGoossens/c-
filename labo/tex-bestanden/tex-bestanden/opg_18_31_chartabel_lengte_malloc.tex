\beginoef
\begin{enumerate}
\item 
Schrijf een functie \verb}lees()} die een lijn inleest vanop het klavier, en een nieuwe c-string teruggeeft. Je weet niet hoe lang de tekst is, maar na oproep van de functie neemt de tekst
niet meer geheugenplaats in dan strikt noodzakelijk. Indien de tekst langer is dan 1000 karaktertekens, breek je het af na het 1000$^e$. Test dit uit door de constante 1000 aan te passen - uiteraard.
\item
Test je functie uit in het onderstaande (half-afgewerkte) hoofdprogramma. 
Vervolledig het hoofdprogramma.
\end{enumerate}
\begin{verbatim}
int main(){
    int i;
    for(i=0; i<5; i++){    
        char * tekst = lees();
        printf("Ik las in %s.\n",tekst);
    }	
    return 0;
}
\end{verbatim}

\endoef

